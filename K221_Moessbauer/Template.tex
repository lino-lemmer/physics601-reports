\documentclass[11pt, english, fleqn, DIV=15, headinclude, BCOR=2cm]{scrreprt}

\usepackage[
    color,
    bibatend,
]{../../header}

\graphicspath{{_build/}}

\hypersetup{
    pdftitle=
}

\subject{Lab report}
\title{Mößbauer effect}
\subtitle{Experiment K221 -- Universität Bonn}
\author{%
    Martin Ueding \\
    \small{\href{mailto:mu@martin-ueding.de}{mu@martin-ueding.de}}
    \and
    Lino Lemmer \\
    \small{\href{mailto:l2@uni-bonn.de}{l2@uni-bonn.de}}
}

\date{2016-03-02}

\publishers{Tutor: Peter Klassen}

\begin{document}

\maketitle

\chapter*{Permission to upload}

I, Martin Ueding, would like to scan and upload this lab report with your
corrections to my website \href{http://martin-ueding.de}{martin-ueding.de}.
There, the original lab report as well as the reviewed one will be licensed
under the “\href{http://creativecommons.org/licenses/by-sa/4.0/}{Creative
Commons Attribution-ShareAlike 4.0 International License}”. Is that okay with
you?

Yes $\Box$ \hspace{2cm} No $\Box$

\begin{abstract}
\end{abstract}

\tableofcontents

\chapter{Theory}

\section{Absorption and resonances}

\begin{figure}
    \centering
    \includegraphics{co57}
    \caption{%
        Decay scheme of $^{57}\mathrm{Co}$ into $^{57}\mathrm{Fe}$. Cobalt from
        the source
        undergoes electron capture (EC) with a very long halftime. The
        transition between to the groundstate is the Mößbauer energy level we
        are interested in.
        %
        Figure adapted from
        \textcite[Abb.~4.8]{Schatz/Nukleare_Festkoerperphysik}.
    }
    \label{fig:co57}
\end{figure}

\end{document}

% vim: spell spelllang=en tw=79
