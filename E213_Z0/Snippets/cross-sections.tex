\begin{figure}
    \centering
    \begin{subfigure}[c]{0.48\linewidth}
        \centering
        \includegraphics{cross-sections-electrons}
        \caption{%
            Electrons
        }
        \label{fig:cross-sections-zoom/electrons}
    \end{subfigure}
    \hfill
    \begin{subfigure}[c]{0.48\linewidth}
        \centering
        \includegraphics{cross-sections-muons}
        \caption{%
            Muons
        }
        \label{fig:cross-sections-zoom/muons}
    \end{subfigure}

    \vspace{2ex}

    \begin{subfigure}[c]{0.48\linewidth}
        \centering
        \includegraphics{cross-sections-taus}
        \caption{%
            Taus
        }
        \label{fig:cross-sections-zoom/taus}
    \end{subfigure}
    \hfill
    \begin{subfigure}[c]{0.48\linewidth}
        \centering
        \includegraphics{cross-sections-hadrons}
        \caption{%
            Hadrons
        }
        \label{fig:cross-sections-zoom/hadrons}
    \end{subfigure}

    \caption{%
        Zoom-in of Figure~\ref{fig:cross-sections}. One can see the different
        qualities of the data points and the fits.
    }
    \label{fig:cross-sections-zoom}
\end{figure}
