\documentclass[11pt, english, fleqn, DIV=15, headinclude, BCOR=2cm]{scrreprt}

\usepackage[
    color,
    bibatend,
]{../../header}

\graphicspath{{./}{../Figures/}}

\usepackage{needspace}

\usepackage{mathtools}
\usepackage{listings}

\lstset{
    basicstyle=\small\ttfamily,
}

\hypersetup{
    pdftitle=
}

\usepackage{longtable}
\usepackage{subcaption}

\usepackage[all]{nowidow}

\subject{Lab report}
\title{Magneto-optical trap}
\subtitle{Experiment A248 -- Universität Bonn}
\author{%
    Martin Ueding \\
    \small{\href{mailto:mu@martin-ueding.de}{mu@martin-ueding.de}}
    \and
    Lino Lemmer \\
    \small{\href{mailto:l2@uni-bonn.de}{l2@uni-bonn.de}}
}

\date{\daterange{2016-04-25}{2016-04-26}}

\publishers{Tutor: } % TODO

\begin{document}

\maketitle

\begin{abstract}
    % TODO
\end{abstract}

\tableofcontents

\chapter*{Permission to upload}

I, Martin Ueding, would like to scan and upload this lab report with your
corrections to my website \href{http://martin-ueding.de}{martin-ueding.de}.
There, the original lab report as well as the reviewed one will be licensed
under the “\href{http://creativecommons.org/licenses/by-sa/4.0/}{Creative
Commons Attribution-ShareAlike 4.0 International License}”. Is that okay with
you?

Yes $\Box$ \hspace{2cm} No $\Box$

\chapter{Theory}

\section{Optical cooling}

\subsection{Radiation pressure}

Light, being mediated by massless gauge bosons, has a quadratic dispersion
relation $E = cp$ with momentum $p = \hbar k$ and wave number $k$. An atom
which absorbs a photon will obtain its energy and also its momentum.

% TODO Why is the emission isotropic?

\subsection{Doppler shift}

\subsection{Red detuning}

\section{Optical molasses}

\subsection{Counterpropagating beams}

\subsection{Cooling but no caught atoms}

\subsection{Doppler temperature}

\section{Magneto-optical trap}

\subsection{Magnetic quadrupole field}

\subsection{Circularly polarized beams}

\subsection{Position depending force}

\section{Rubidium}

\subsection{Level structure}

\subsection{Non-resonant excitation}

\subsection{Inelastic collisions}

\subsection{Dark states}

\subsection{Repumping beam}

\section{Equipment}

\subsection{Anti-Helmholtz coils}

\subsection{Vacuum chamber}

\subsection{Laser system}

\subsection{Diode lasers}

\section{Doppler-free spectroscopy}

\subsection{Pump and signal beam}

\subsection{Groups of atoms with similar velocity}

\subsection{Lamp dip}

\subsection{Crossover resonance}

\subsection{Rubidium spectrum}

\section{Polarization spectroscopy}

\subsection{Circularly polarized light}

\subsection{Anisotropic pumping}

\subsection{Birefringence}

\subsection{Detection of tilt}

\subsection{Dispersion}

\subsection{Kramers-Kronig relation}

\chapter{Conduction}

\section{Apparatus}

\section{Setup and calibration}

Locking laser on cooling transition.

Light through fiber, \SI{16.8}{\milli\watt}.

Turn off re-pumper (paper in front) \SI{14.1}{\milli\watt}

Adjust fiber such that PBS is hit, also bottom elevator mirror is hit nicely.

Tweak bottom $z$-mirror such that half of the ray is covered when using the
black sheet. Then adjust the other axis with the folded piece of cardboard.

Next we block the elevator. Then adjust the top mirror such that the beams
coincide.

Measure fiber power again, \SI{15.1}{\milli\watt}. Set the lower part to
\SI{5.0}{\milli\watt} via the $\lambda/2$ plate behind the fiber. Then we have
no power on the elevator. Adjust again to have \SI{8.8}{\milli\watt} on the
elevator and \SI{4.8}{\milli\watt} on the $z$-axis.

We realize that we incur loss on all the mirrors and first adjust the splitting
in the transversal ($x$ and $y$) direction to be symmetric. There are
\SI{2.7}{\milli\watt} on each of them. Now we adjust the initial $\lambda/2$
plate to give more power to the transversal and less to the longitudinal
direction. We then have \SI{3.1}{\milli\watt} behind each polarizing beam
splitter.

Start to adjust transversal mirrors. We calibrate the $y$ direction first. The
reflected beam coincides nicely with the incident beam. Then we move on to the
$x$ direction and expect that we will have to further work on the $y$
direction. 

Then we re-lock the cooling laser to the transition. We scan through the
pumping laser spectrum and set it to the re-pumping transition.

We tweak on all the mirrors to optimize the overlap. When toggling the power
supply of the magnetic field, we can see a little difference in the brightness.

% TODO Insert both images, as well as the difference, into here.

Adjusting the laser locks made the MOT even larger.

Although the manual has it later, we took a few shots at the loading behavior.

Replacing the photo diode with the power meter, we see that out MOT is very
bright. With underground, it has around \SI{560}{\nano\watt}.

% TODO Add table with measurements.

Lens says something with \SI{50}{\milli\meter}. Its distance to the MOT are
\SI{10}{\centi\meter}, roughly estimated.

Pressure is supposed to be \SIrange{1.1e-7}{1.5e-7}{\milli\bar}

$x$: \SI{3.57}{\milli\watt}
$y$: \SI{3.36}{\milli\watt}
$z$: \SI{4.45}{\milli\watt}

\SI{3.9}{\milli\watt}

The effect of the $\lambda/4$ plates in front of the MOT have a significant
effect. One can loose the MOT completely. The effect of the plates in the
backreflected beam is not very large. Ideally, this should be zero, of
course.

Go through the magnetic field current.

% TODO Tablet with magnetic field.

Room temperature \SI{24}{\celsius}

Unlock the cooling laser. Run a slow frequency of around \SI{100}{\milli\hertz}
and scan the MOT intensity with the photo diode. 

\end{document}

% vim: spell spelllang=en_us tw=79
