\documentclass[11pt, english, fleqn, DIV=15, headinclude, BCOR=2cm]{scrreprt}

\usepackage[
    color,
    bibatend,
]{../../header}

\graphicspath{{./}{../Figures/}}

\usepackage{needspace}

\usepackage{mathtools}
\usepackage{listings}

\lstset{
    basicstyle=\small\ttfamily,
}

\hypersetup{
    pdftitle=
}

\usepackage{longtable}
\usepackage{subcaption}

\usepackage[all]{nowidow}

\subject{Lab report}
\title{Magneto-optical trap}
\subtitle{Experiment A248 -- Universität Bonn}
\author{%
    Martin Ueding \\
    \small{\href{mailto:mu@martin-ueding.de}{mu@martin-ueding.de}}
    \and
    Lino Lemmer \\
    \small{\href{mailto:l2@uni-bonn.de}{l2@uni-bonn.de}}
}

\date{\daterange{2016-04-25}{2016-04-26}}

\publishers{Tutor: Daniel Babik}

\begin{document}

\maketitle

\begin{abstract}
    In this experiment we set up a magneto-optical trap to catch rubidium
    atoms. After that we investigate basic properties of the trap and its
    components.
\end{abstract}

\tableofcontents

\chapter*{Permission to upload}

I, Martin Ueding, would like to scan and upload this lab report with your
corrections to my website \href{http://martin-ueding.de}{martin-ueding.de}.
There, the original lab report as well as the reviewed one will be licensed
under the “\href{http://creativecommons.org/licenses/by-sa/4.0/}{Creative
Commons Attribution-ShareAlike 4.0 International License}”. Is that okay with
you?

Yes $\Box$ \hspace{2cm} No $\Box$

\chapter{Theory}

\section{Optical cooling}

\subsection{Radiation pressure}

Light, being mediated by massless gauge bosons, has a quadratic dispersion
relation $E = cp$ with momentum $p = \hbar k$ and wave number $k$. An atom
which absorbs a photon will obtain its energy and also its momentum. While in
stimulated emission the momentum of the emitted photon points in the same
direction as the stimulating atom's momentum, spontaneous emission is isotropic
which leads to a net acceleration. This effect causes the so called radiation pressure.

\subsection{Red detuning}

To cool atoms down one has to make sure that only those atoms that move
towards the laser beam feel this pressure. This can be done using the Doppler
effect: In the rest system of the atom, that flies towards the laser beam, the
photons of the laser are blue detuned, with $\nu'=\nu(1+v/c)$, where $v$ is
the velocity of the atom. To get those atoms on resonance again one has to
detune the laser to the red.

\subsection{Optical molasses}

The force of a laser with intensity $I$ and detuning $\delta =
\omega_\text{laser}-\omega_\text{res}$ on an atom can be expressed as
\[
    F_\text{radiation} = \frac{\hbar k\Gamma}2\cdot\frac{I/I_\text{S}}{\del{2\frac{\delta -
    kv}\Gamma}^2 + 1 + I/I_\text{S}}\, ,
\]
where $k$ denotes the wave vector of the photons, $\Gamma$ the decay rate of the
excited state and $I_\text{S}$ the saturation intensity. This force is non zero
at $v=0$ as we would need to cool down. To solve this one uses two
counterpropagating laser beams with the same detuning. For small velocities
this gives a frictional force:
\[
    F_\text{count} = \frac{8\hbar k^2\delta}{\Gamma}
    \frac{I/I_\text{S}}{\del{\del{2\delta/\Gamma}^2 + 1 + I/I_\text{S}}^2}
    \cdot v \underset{\delta<0}{=} -\alpha\cdot v \, .
\]
Applying three orthogonal pairs of lasers gives an effective deacceleration of
atoms and with this a cooling. This setup is known as \emph{optical molasses}.
This process is limited by heating due to spontaneous emission. This leads to
the so called \emph{Doppler limit}:
\[
    T_\text{Doppler} = \frac{\hbar \Gamma}{2k_\text{B}} \, .
\]

\section{Magneto-optical trap}

Since the forces in the optical molasses are independent of space, the atoms
have the possibility to diffuse out of the cooling area. Due to collisions
those atoms get heated up again. We need an additional force that keeps the
atoms inside that area. One possibility is to use a \emph{magneto optical trap}
(\textsc{mot}).

To simplify the working principle we consider a two level system with one
possible transition ($F=0 \to F=1$). Also we add a 1-dimensional linear
increasing magnetic field (w.l.o.g.) along the $z$ axis, which is zero at
$z=0$. Due to the Zeeman effect we get a space-dependent energy splitting of
the three possible $F=1$ configurations, which causes a lowering of the $m=1$
state's energy for $z<0$ and a lowering of the $m=-1$ level for $z>0$ (see
Figure~\ref{fig:mot-principle}).

Now we add two counterpropagating laser beams with detuning $\delta<0$. If now
the laser beam propagating in positive $z$ direction has a $\sigma^+$ helicity
and the counterpropagating beam has a $\sigma^-$ helicity, atoms which are
located at $z<0$ have a higher possibility to absorb $\sigma^+$ photons and
hence get pushed back to $z=0$. For atoms at $z>0$ it is the other way around.
There the absorption of a $\sigma^-$ photon is more likely, so that they 
also get pushed back to $z=0$.

While in reality the energy levels are way more complex the working principle
remains the same.

\begin{figure}
    \centering
    \includegraphics[width=.5\textwidth]{mot-principle}
    \caption{%
        Energy levels and transitions in a spatially varying magnetic field.
    }
    \label{fig:mot-principle}
\end{figure}

\section{Rubidium}

\subsection{Level structure}

\subsection{Non-resonant excitation}

\subsection{Inelastic collisions}

\subsection{Dark states}

\subsection{Repumping beam}

\section{Equipment}

\subsection{Anti-Helmholtz coils}

\subsection{Vacuum chamber}

\subsection{Laser system}

\subsection{Diode lasers}

\section{Doppler-free spectroscopy}

\subsection{Pump and signal beam}

\subsection{Groups of atoms with similar velocity}

\subsection{Lamp dip}

\subsection{Crossover resonance}

\subsection{Rubidium spectrum}

\section{Polarization spectroscopy}

\subsection{Circularly polarized light}

\subsection{Anisotropic pumping}

\subsection{Birefringence}

\subsection{Detection of tilt}

\subsection{Dispersion}

\subsection{Kramers-Kronig relation}

\chapter{Conduction}

\section{Apparatus}

\section{Setup and calibration}

Locking laser on cooling transition.

Light through fiber, \SI{16.8}{\milli\watt}.

Turn off re-pumper (paper in front) \SI{14.1}{\milli\watt}

Adjust fiber such that PBS is hit, also bottom elevator mirror is hit nicely.

Tweak bottom $z$-mirror such that half of the ray is covered when using the
black sheet. Then adjust the other axis with the folded piece of cardboard.

Next we block the elevator. Then adjust the top mirror such that the beams
coincide.

Measure fiber power again, \SI{15.1}{\milli\watt}. Set the lower part to
\SI{5.0}{\milli\watt} via the $\lambda/2$ plate behind the fiber. Then we have
no power on the elevator. Adjust again to have \SI{8.8}{\milli\watt} on the
elevator and \SI{4.8}{\milli\watt} on the $z$-axis.

We realize that we incur loss on all the mirrors and first adjust the splitting
in the transversal ($x$ and $y$) direction to be symmetric. There are
\SI{2.7}{\milli\watt} on each of them. Now we adjust the initial $\lambda/2$
plate to give more power to the transversal and less to the longitudinal
direction. We then have \SI{3.1}{\milli\watt} behind each polarizing beam
splitter.

Start to adjust transversal mirrors. We calibrate the $y$ direction first. The
reflected beam coincides nicely with the incident beam. Then we move on to the
$x$ direction and expect that we will have to further work on the $y$
direction. 

Then we re-lock the cooling laser to the transition. We scan through the
pumping laser spectrum and set it to the re-pumping transition.

We tweak on all the mirrors to optimize the overlap. When toggling the power
supply of the magnetic field, we can see a little difference in the brightness.

% TODO Insert both images, as well as the difference, into here.

Adjusting the laser locks made the MOT even larger.

Although the manual has it later, we took a few shots at the loading behavior.

Replacing the photo diode with the power meter, we see that out MOT is very
bright. With underground, it has around \SI{560}{\nano\watt}.

% TODO Add table with measurements.

Lens says something with \SI{50}{\milli\meter}. Its distance to the MOT are
\SI{10}{\centi\meter}, roughly estimated.

Pressure is supposed to be \SIrange{1.1e-7}{1.5e-7}{\milli\bar}

$x$: \SI{3.57}{\milli\watt}
$y$: \SI{3.36}{\milli\watt}
$z$: \SI{4.45}{\milli\watt}

\SI{3.9}{\milli\watt}

The effect of the $\lambda/4$ plates in front of the MOT have a significant
effect. One can loose the MOT completely. The effect of the plates in the
backreflected beam is not very large. Ideally, this should be zero, of
course.

Go through the magnetic field current.

% TODO Tablet with magnetic field.

Room temperature \SI{24}{\celsius}

Unlock the cooling laser. Run a slow frequency of around \SI{100}{\milli\hertz}
and scan the MOT intensity with the photo diode. 

\end{document}

% vim: spell spelllang=en_us tw=79
